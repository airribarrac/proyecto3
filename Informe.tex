\documentclass[12pt]{article}
\usepackage{graphicx}
\usepackage{float}
\usepackage{multirow}
\usepackage{blindtext}
\usepackage{algpseudocode}
\linespread{1.2}
\graphicspath{{Pictures/}}
\usepackage[utf8]{inputenc}
\renewcommand{\contentsname}{Índice}

\begin{document}
\begin{titlepage}

\newcommand{\HRule}{\rule{\linewidth}{0.5mm}}

\center

\textsc{\LARGE Universidad de Concepci\'on}\\[1.5cm] 
\textsc{\Large Ingenier\'ia Civ\'il Inform\'atica}\\[0.5cm]
\textsc{\large Estructura de Datos}\\[2.5cm]

\HRule \\[0.4cm]
{ \huge \bfseries Proyecto 3}\\[0.4cm]
\HRule \\[3.5cm]

\textsc{\large Alexander Irribarra, Angelo Zapata}\\[0.5cm]
\textsc{Profesor Diego Seco}\\
\textsc{Ayudantes: Diego Gatica, Paulo Olivares}\\[0.5cm]
{\large 23 de Junio, 2017}\\[3cm]

\vfill 
\end{titlepage}
\tableofcontents
\newpage
\section{Introducción}
\indent\indent  Una red social es un sistema de relaciones entre individuos de una sociedad o grupo, compuesta por un conjunto de actores, que pueden ser individuos personales u organizaciones, que estan relacionados de acuerdo a algun criterio.

El uso de las redes sociales en internet, en la actualidad, es algo muy importante para entablar relaciones entre individuos que compartan algún interes en comun o alguna relación en algún ambito de sus vidas.

Para este proyecto se construirá una mini red social, muy similar al conocido Twitter, en el cuál la relación entre las personas será a qué personas siguen (Follow), ésto se implementará de una forma abstracta con un grafo y con sus correspondientes métodos para que sea eficiente y fácil de entender.
\newpage
\section{Desarrollo}

\indent\indent Se construyó un diGraphADT para representar la red social con los siguientes métodos importantes, con su respectiva explicación de cómo se implementó.

\subsection{Find(u)}

\indent \indent Para elegir la mejor implementación de este método, se comparó una forma lineal, recorriendo todos los nodos del grafo (con una complejidad O(n+m), puesto que usa un DFS) y una implementada con un ADT Map de la STL de C++, dando como mejor resultado la implementacion con el ADT Map, el cual la complejidad es de O(log n) puesto que es un arbol binario de busqueda y se decidió por esa implementación por la garantía de peor caso (en comparacion de unsorted map implementado con tablas hash), ya que una red social es de un tamaño considerable.

Al llamar a este método, retorna si el usuario u existe en el grafo o no.

\subsection{Cliqué()}

\indent\indent Un cliqué es un subgrafo en que cada vértice está conectado a cada otro vértice del subgrafo, es decir, todos los vértices del subgrafo son adyacentes formando un grafo completo.

Para implementar este método, se uso una solución conocida por Bron-Kerbosch para grafos no dirigidos, pero adaptada a grafos dirigidos. Esta consiste en un método privado llamado BK(R,P,X) el cual trabaja con 3 subconjuntos de nodos R, P y X los cuales en la primera llamada P es el conjunto de nodos, R y X son conjuntos vacíos, luego se itera entre ellos y cuando P y X son conjuntos disjuntos, el R es un cliqué. Asi P y X son conjuntos disjuntos cuya unión son aquellos vertices que forman cliqué cuando son añadidos a R.

Al llamar a este método, nos imprime los nodos de los cliqués maximales. 
\subsection{Compact()}

\indent\indent Al implementar este método, se usó de base el cliqué descrito anteriormente para obtener los nodos a compactar. Después, solo se consideran cliqués disjuntos pues se marcan los nodos que ya están en un cliqué anterior, si el cliqué sin los nodos anteriores sigue sirviendo, se considera para hacer un nodo virtual y se le da un nombre e indice para una matriz de adyacencia, una vez que se repasa todos los cliqués, continua con los nodos que no quedaron marcados y se sigue mapeando de la misma forma dandoles nombre e indice. 

En este punto ya se tiene mapeado cada nodo original a uno del grafo compacto, entonces se crea una matriz de adyacencia y se marcan las aristas del grafo original a la con los nodos transformados, luego solo se recorre la matriz imprimiendo los pares.

Al llamar a este método, compacta los cliqués en nodos virtuales compactos.

\subsection{Follow(n)}

Éste método sugiere los n usuarios más importantes de la red.
Para obtener esto, se recorren los nodos y se almacenan en una Priority Queue por el criterio más importante, que es el mayor grado de llegada del nodo (inDeg), luego por el menor grado de salida (outDeg) y finalmente por orden alfabético.

Al llamar a este método, nos imprime los n usuarios más importantes (o más populares).

\newpage
\section{Conclusión}
\indent \indent Para finalizar el presente proyecto, se puede concluir que una red social es una estructura compleja con una gran cantidad de usuarios conectados entre sí, además de que es fácil de comprender a grandes razgos su funcionamiento y funcionan de forma rápida, por lo cual son tan populares.

Para hacerlo mas sencillo se ideó que los grafos son capaces de modelar una red social conectada, con los nodos como los usuarios y las arista las conexiones entre ellos.

En este mini Twitter implementado, se usaron cuatro métodos principales, el find(u) que simplemente nos dice si el usuario existe o no, el cliqué() que nos dice que usuarios estan completamente conectados con otros (podria servir como posterior análisis, concluyendo si son compañeros de curso que por lo general se conectan todos con todos), el compact() que le da eficiencia al grafo haciendolo más "pequeño" la conexión entre usuarios y por ultimo el follow(n) que nos entregan los usuarios más importantes o populares de la red social (perfectos para hacer alguna difusión).

Todo esto se implementó tanto con algoritmos propios como con otros conocidos, para que sean más eficientes.

\end{document}
